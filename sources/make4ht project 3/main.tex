\documentclass[twoside,a4paper]{article}

\setlength{\oddsidemargin}{0in}
\setlength{\evensidemargin}{0in}
\setlength{\topmargin}{-0.6in}
\setlength{\textwidth}{\paperwidth}
\addtolength{\textwidth}{-2in}
\setlength{\textheight}{\paperheight}
\addtolength{\textheight}{-2in}

\usepackage{amsmath,amsthm,amssymb,latexsym}
\usepackage[v2,tips]{xy}

\newtheorem{theorem}{Theorem}[section]
\newtheorem{proposition}[theorem]{Proposition}
\newtheorem{corollary}[theorem]{Corollary}
\newtheorem{lemma}[theorem]{Lemma}

\theoremstyle{definition}
\newtheorem{definition}[theorem]{Definition}
\newtheorem{example}[theorem]{Example}

\theoremstyle{remark}
\newtheorem{remark}[theorem]{Remark}

\newcommand{\ap}{{\operatorname{AP}}}
\newcommand{\AP}{{\mathbb{AP}}}
\newcommand{\Trig}{{\operatorname{Trig}}}
\newcommand{\ad}{{\operatorname{ad}}}
\newcommand{\hip}[2]{(#1|#2)}
\newcommand{\inten}{\check\otimes}
\newcommand{\mc}{\mathcal}
\newcommand{\mf}{\mathfrak}
\newcommand{\id}{\operatorname{id}}
\newcommand{\ip}[2]{\langle{#1},{#2}\rangle}
\newcommand{\aone}{\Box}
\newcommand{\atwo}{\Diamond}
\newcommand{\proten}{{\widehat{\otimes}}}

\begin{document}

\large
\title{Quantum compactifications of Fourier Algebras}
\author{Matthew Daws}
\maketitle

\begin{abstract}
To write.

2000 \emph{Mathematics Subject Classification:}
   Primary ???;
   Secondary ???

\emph{Keywords:} ???
\end{abstract}

\section{Introduction}

The classical Bohr Compactification of a topological (semi)group
can be defined in terms of unitary representations (see the
original paper of von Neumann \cite{vn1} or see
\cite[???]{BJM} for a modern treatment).
Recently, see \cite{soltan}, So{\l}tan has extended this idea to give
a definition of a compactification of a quantum semigroup, a
process which yields a compact quantum group, in the sense of
Woronowicz (see \cite{woro1}).

For a locally compact group $G$, we can describe the Bohr (or
equivalently \emph{almost periodic}) compactification of $G$
in terms of a certain function space on $G$.  Indeed, we can describe
this function space in terms of the Banach algebra $L^1(G)$ and
its dual $L^\infty(G)$.  We turn the dual into an $L^1(G)$-bimodule
in the usual way, and then define $f\in L^\infty(G)$ to be
\emph{almost periodic} if the map
\[ R_f:L^1(G)\rightarrow L^\infty(G); \quad
a\mapsto a\cdot f \qquad (a\in L^1(G)) \]
is compact.  If we let $\ap(G)\subseteq L^\infty(G)$ be the collection
of almost periodic elements, then each member of $\ap(G)$ is
continuous; $\ap(G)$ is a unital C$^*$-subalgebra of $L^\infty(G)$;
if the character space of $\ap(G)$ is $G^\ap$, then $G^\ap$ becomes
a compact group containing a dense homeomorphic image of $G$.

For an abelian locally compact group $G$, we have the Fourier
transform $L^1(G)\rightarrow C_0(\hat G)$, where $\hat G$ is the
dual group to $G$.  The image is defined as the \emph{Fourier algebra}
of $\hat G$, see \cite[???]{Dales}.  For a non-abelian group $G$, Eymard
defined an commutative Banach algebra of functions on $G$, written
$A(G)$, which extends the definition of the Fourier algebra, see \cite{eymard}.
As the definition of \emph{almost periodic} above makes sense for any
Banach algebra, it is not surprising that the space of almost
periodic functionals on the Fourier algebra has been studied.
The definition appears to have been made in \cite[Section~7]{DR},
where the authors write $\ap(\hat G)$, in analogy with the abelian case.
Further study was undertaken in \cite{gran1} and \cite{lau}, for example.
An excellent survey (which we use extensively below) is Chou's paper
\cite{chou}.  See also \cite{hu} and \cite{must} for more recent work.

The space $\ap(\hat G)$ is not nearly as well understood as $\ap(G)$,
and it appears that no study has been undertaken of $\ap(\hat G)$ in
the context of a categorical notion of ``compactification''.  The aim
of this paper is to undertake such a study.

\textbf{Stuff about approximation properties}.

\textbf{AIMS:}
\begin{itemize}
\item Heavily using \cite{chou}, we introduce a modified definition
of ``almost periodic'', and show the resulting space exactly
coincides with the quantum compactification constructed by
So{\l}tan, once appropriate identifications are made.
\item We try to use operator-space methods.
\item In general $\ap(\hat G)$ is not a C$^*$-algebra; this is our major
problem with what we could call the ``classical case''.  However, does there
perhaps exist a \emph{completely isometry} $\Delta:\ap(\hat G)\rightarrow
\ap(\hat G)\inten\ap(\hat G)$.
\item Proceeding with the above, can we apply some abstract procedure to
$\ap(\hat G)$ to obtain a von Neumann algebra?  Use the Haar functional here
(which we know does exist).
\item What is the \emph{reduced} compact quantum group which we get from
So{\l}tan's methods?  Can we perhaps recover this from $\ap(\hat G)$?
\item This about the tensor product: key fact which Soltan uses to show
that $\AP(\mathbb G)$ is an algebra.  Can we adapt to $\ap(\hat G)$ somehow?
\item Anything else?
\end{itemize}

Notation: $\mc B(E)$.  Dual space, dual pairing.  Notation for inner products.






\section{Compact quantum group compactifications}

We shall quickly sketch So\l tan's work in \cite{soltan}.
Following So\l tan, we shall define a \emph{compact quantum semigroup}
to be a pair $\mathbb S=(\mc A,\Phi)$ where $\mc A$ is a unital C$^*$-algebra
and $\Phi:\mc A\rightarrow\mc A\otimes\mc A$ is a \emph{co-associative
product}.  Here, and henceforth, $\mc A\otimes\mc A$ is the \emph{minimal}
or \emph{spacial} tensor product of $\mc A$ with itself.  To say that $\Phi$ is a
co-associative product means that $\Phi$ is a unital $*$-homomorphism and
$(\Phi\otimes\id)\Phi = (\id\otimes\Phi)\Phi$.

A \emph{compact quantum group} is a compact quantum semigroup $\mathbb G=(\mc A,\Phi)$
such that the linear span of the either of the sets
$\{(a\otimes I)\Phi(b) : a,b\in\mc A\}$ or $\{(I\otimes a)\Phi(b) : a,b\in\mc A\}$
is dense in $\mc A\otimes\mc A$.  These density conditions imply a large
amount of structure: see \cite{woro1} or the survey \cite{maes} for further
details. The density conditions can actually be weakened: see \cite{murphy}.

Compact quantum groups have a very rich (co)representation theory.  For example,
every irreducible corepresentation is finite-dimensional.  As in the classical case,
S{\o}ltan's quantum compactification is stated in terms of finite-dimensional
corepresentations.  We first need the notion of a \emph{quantum semigroup}.
This is a pair $\mathbb S=(\mc A,\Phi)$ where $\mc A$ is a C$^*$-algebra (now not
assumed unital) and $\Phi:\mc A\rightarrow M(\mc A\otimes\mc A)$ is a co-associative,
non-degenerate $*$-homomorphism.  Here $M(\mc A\otimes\mc A)$ is the \emph{multiplier
algebra} of $\mc A\otimes\mc A$.  See \cite{lance}, for example, for further details
about multiplier algebras.  Recall that to say that $\Phi$ is \emph{non-degenerate} means
that $\Phi(\mc A)(\mc A\otimes\mc A)$ is dense in $\mc A\otimes\mc A$.  As such, there
is a unique extension of $\Phi$ to a $*$-homomorphism $M(\mc A)\rightarrow M(\mc A\otimes
\mc A)$.  Similarly, $\Phi\otimes\id$ and $\id\otimes\Phi$ make sense as maps
$M(\mc A\otimes\mc A)\rightarrow M(\mc A\otimes\mc A\otimes\mc A)$, and so it
makes sense to speak of $\Phi$ as being co-associative.

We now state the definitions which S{\o}ltan makes in \cite{soltan}.
However, we shall state them in different ways which are more compatible with
our Banach algebra approach.  Fix a quantum semigroup $\mathbb S=(\mc A,\Phi)$.
In the usual way, we use $\Phi$ to induce an associative product on $\mc A'$,
turning $\mc A'$ into a Banach algebra, by
\[ \ip{\mu\lambda}{a} = \ip{\mu\otimes\lambda}{\Phi(a)}
\qquad (a\in\mc A, \mu,\lambda\in\mc A'). \]

Let $X$ be a finite-dimensional Banach space, and let $T\in\mc B(X)\otimes M(\mc A)$.
For each $\mu\in\mc A'$, let $T_\mu\in\mc B(X)$ be the slice map,
$T_\mu = (\id\otimes\mu)T$.  We say that $T$ is a \emph{bounded representation} if
$T$ is invertible in the algebra $\mc B(X)\otimes M(\mc A)$, and the map
$\mc A'\rightarrow\mc B(X), \mu\mapsto T_\mu$ is an algebra homomorphism.
Notice that we always have the \emph{trivial representation}, $I_X\otimes I_{\mc A}$.
It is easy to construct direct sums of bounded representations, and with a little
more effort, tensor products, see \cite[Section~2.1]{soltan}.

Let $\ad:\mc B(X)\rightarrow\mc B(X'), S\mapsto S'$ be the \emph{adjoint map}.
For $T\in\mc B(X)\otimes M(\mc A)$, let $T^T = (\ad\otimes\id)T$ be the
\emph{transpose} of $T$.  Then a bounded representation $T$ is \emph{admissible}
if $T^T$ is invertible in $\mc B(X')\otimes M(\mc A)$.

When $H$ is a (finite-dimensional) Hilbert space, $\mc B(H)\otimes M(\mc A)$ becomes
a C$^*$-algebra.  When a bounded representation $T\in\mc B(H)\otimes M(\mc A)$
is a unitary, we say that $T$ is a \emph{unitary representation}.  We say that a bounded
representation $T\in\mc B(X)\otimes M(\mc A)$ is \emph{similar to a unitary
representation} if $X$ admits a Hilbert space structure turning $T$ into a unitary
representation.  Then S{\o}ltan shows that admissible representations are similar
to unitary representations; the converse need not hold.

Let $T\in\mc B(X)\otimes M(\mc A)$ be an admissible representation, and let
$\lambda\in\mc B(X)'$.  Then $(\lambda\otimes\id)T\in M(\mc A)$ is a \emph{matrix
element of $T$}.  S{\o}ltan shows that the linear span of matrix elements of
all admissible representations is a unital $*$-subalgebra of $M(\mc A)$, see
\cite[Proposition~2.12]{soltan}.  Let $\AP(\mathbb S)$ be the closure of this
algebra, so $\AP(\mathbb S)\subseteq M(\mc A)$ is a unital C$^*$-algebra.  Let
$\Delta_{\AP(\mathbb S)}$ be the restriction of $\Phi$ to $\AP(\mathbb S)$.
Then $\Delta_{\AP(\mathbb S)}$ is a $*$-homomorphism from $\AP(\mathbb S)$ to
$\AP(\mathbb S)\otimes\AP(\mathbb S)$, and the pair $\mf b\mathbb S =
(\AP(\mathbb S),\Delta_{\AP(\mathbb S)})$ satisfies the density conditions
to become a compact quantum group.

The compact quantum group $\mf b\mathbb S$ satisfies the expected universal
property.  Let $\chi_{\mathbb S}:\AP(\mathbb S)\rightarrow M(\mc A)$ be the
inclusion map.  Then, if $\mathbb G=(\mc B,\Phi_\mc B)$ is a compact quantum group,
and $\theta:\mc B\rightarrow M(\mc A)$ is a non-degenerate\footnote{As $\mc B$ is
unital, this is equivalent to $\theta$ being a unital $*$-homomorphism} $*$-homomorphism such
that $\Phi\circ\theta = (\theta\otimes\theta)\circ \Phi_{\mc B}$, there is a unique
$*$-homomorphism $\mf b\theta:\mc B\rightarrow\AP(\mathbb S)$ such that
$\chi_{\mathbb S} \circ \mf b\theta = \theta$.

If $S$ is a topological semigroup, $\mc B=C_0(S)$, and $\Phi$ is the canonical
dualisation of the product on $S$, then $\AP(\mathbb S)$ is isomorphic to
$C(S^\ap)$, where $S^\ap$ is the classical Bohr compactification of $S$.


\subsection{For the Fourier algebra}

We now turn our attention to the Fourier algebra $A(G)$.  Let $G$ be a locally compact
group.  Let $\lambda:G\rightarrow\mc B(L^2(G))$ be the left-regular representation
\[ (\lambda(s)f)(t) = f(s^{-1}t) \qquad (f\in L^2(G), s,t\in G). \]
By integrating, we get a contractive homomorphism $\lambda:L^1(G)\rightarrow\mc B(L^2(G))$.
The norm closure of $\lambda(L^1(G))$ is $C^*_r(G)$, the \emph{reduced group C$^*$-algebra
of $G$}.  If we take the weak-operator topology closure, then we get $VN(G)$, the
\emph{group von Neumann algebra}.  Equivalently, $VN(G)$ is the von Neumann algebra
generated by $\{ \lambda(s) : s\in G \}$.
The predual of $VN(G)$ is $A(G)$, the \emph{Fourier
algebra of $G$}.  This algebra is in standard position on $L^2(G)$, and so, for each
$a\in A(G)$, there exists $f,g\in L^2(G)$ such that
\[ \ip{T}{a} = \hip{Tf}{g} \qquad (T\in VN(G)). \]
We can treat $a$ as a member of $C_0(G)$ by
\[ a(s) = \ip{\lambda(s)}{a} = \hip{\lambda(s)f}{g}
= \int_G f(s^{-1}t) \overline{g(t)} \ dt \qquad (s\in G). \]

We identify $L^2(G)\otimes L^2(G)$ with $L^2(G\times G)$.  Define a unitary
$W\in\mc B(L^2(G\times G)$ by
\[ (Wf)(s,t) = f(s,st) \qquad (s,t\in G, f\in L^2(G\times G)). \]
Then we may define a map $\Delta:VN(G)\rightarrow VN(G)\overline\otimes VN(G)
= VN(G\times G)$ by
\[ \Delta(T) = W^*(T\otimes I)W \qquad (T\in VN(G)). \]
We can check that $\Delta(\lambda(s)) = \lambda(s)\otimes\lambda(s) = \lambda(s,s)$
for $s\in G$.  Then $\Delta$ is a normal $*$-homomorphism which, in the appropriate
sense, is co-associative.  The pre-adjoint of $\Delta$ turns $A(G)$ into a Banach
algebra, and we can check that $A(G)$ becomes a (not closed) subalgebra of $C_0(G)$.

There are lots of ways to define the \emph{multiplier} algebra of a C$^*$-algebra.
For us, the most useful approach shall be to follow Pedersen, \cite[Section~3.12]{ped}.
If $\mc A$ is a C$^*$-algebra inside $\mc B(H)$, and $\mc M\subseteq\mc B(H)$ is the von
Neumann algebra generated by $\mc A$, then we can identify $M(\mc A)$ with
\[ \{ x\in\mc M : ax,xa \in \mc A \ (a\in\mc A) \}. \]

Using this, we can identify $M(C^*_r(G))$ as a $*$-subalgebra of $VN(G)$.  By the
definition of the spacial tensor product, we identify $C^*_r(G)\otimes C^*_r(G)$ with
a $*$-subalgebra of $VN(G)\overline\otimes VN(G) = VN(G\times G)$.  Hence,
again, we identify $M(C^*_r(G)\otimes C^*_r(G))$ with a $*$-subalgebra of
$VN(G\times G)$.  We can then show that $\Delta$ restricts to give a coassociative,
non-degenerate $*$-homomorphism $C^*_r(G) \rightarrow M(C^*_r(G) \otimes C^*_r(G))$, and 
so $\mathbb G = (C^*_r(G),\Delta)$ is a quantum semigroup (which obviously also has extra
structure).  It is simple to see that $\lambda(s) \in M(C^*_r(G))$ for each $s\in G$.
For further details, see \cite[Chapter~8]{lance}, although be aware that Lance looks
at the right-regular representation.

Following Chou, \cite{chou}, we define $\Trig(\hat G)$ to be the linear span of
$\{\lambda(s):s\in G\}$ in $M(C^*_r(G))$, and let $C^*_\delta(G)$ be the closure.

\begin{proposition}
We have that $\Delta$ restricts to $C^*_\delta(G)$ to give a $*$-homomorphism
$C^*_\delta(G) \rightarrow C^*_\delta(G)\otimes C^*_\delta(G)$, turning $C^*_\delta(G)$
into a compact quantum group.
\end{proposition}
\begin{proof}
By the definition of the spacial tensor norm, it is clear that $C^*_\delta(G)\otimes
C^*_\delta(G)$ is the closure of $\Trig(\hat G)\otimes\Trig(\hat G)$ inside
$VN(G)\overline\otimes VN(G) = VN(G\times G)$.  It is hence easy to see that
$C^*_\delta(G) \otimes C^*_\delta(G)$ can be identified with $C^*_\delta(G\times G)$,
from which it follows that $\Delta$ does map $C^*_\delta(G)$ into
$C^*_\delta(G) \otimes C^*_\delta(G)$.

Notice that
\[ (\lambda(st^{-1})\otimes I)\Delta(\lambda(t))
= (\lambda(st^{-1})\otimes I)(\lambda(t)\otimes\lambda(t))
= \lambda(s)\otimes\lambda(t) \qquad (s,t\in G). \]
It hence follows that $\{ (a\otimes I)\Delta(b) : a,b\in C^*_\delta(G) \}$ is
linearly dense in $C^*_\delta(G) \otimes C^*_\delta(G) = C^*_\delta(G\times G)$.
Similarly, $\{ (I\otimes a)\Delta(b) : a,b\in C^*_\delta(G) \}$ is
linearly dense in $C^*_\delta(G) \otimes C^*_\delta(G)$, showing that $C^*_\delta(G)$
is a compact quantum group.
\end{proof}

We remark that the Haar state $h$ on $C^*_\delta(G)$ is simply the map
given by $h(\lambda(s)) = 1$ if $s=e_G$ the identity of $G$, and $0$ otherwise.
Let $G_d$ be the group $G$ with the discrete topology.  If $G_d$ is amenable,
then we know that $C^*_r(G_d) = C^*(G_d)$, and so it follows that $C^*_\delta(G)$
is isomorphic to $C^*_r(G_d)$, and in particular, the Haar state is faithful.
In general, it seems possible that the Haar state could fail to be faithful.
\footnote{Can we find an example?  Do we address this later?}

\begin{theorem}
Let $\mathbb G = (C^*_r(G),\Delta)$ be as above.  Then $\mf b\mathbb G = (C^*_\delta(G),\Delta)$.
\end{theorem}
\begin{proof}
By the universal property, we have that $C^*_\delta(G) \subseteq \AP(\mathbb G)$.
We follow S{\o}ltan's construction of $\AP(\mathbb G)$; in particular, we shall show
that if $T\in\mc B(H)\otimes M(C^*_r(G))$ is a unitary representation, then each
matrix element of $T$ is a member of $\Trig(\hat G)$.  This shows that $\AP(\mathbb G)
\subseteq C^*_\delta(G)$, as required.

If $H$ is $n$-dimensional, then we can identify $\mc B(H)$ with $\mathbb M_n$,
and so identify $\mc B(H)\otimes M(C^*_r(G))$ with $\mathbb M_n(M(C^*_r(G))$.
Let $T=(T_{ij})_{1\leq i,j\leq n}$.  Then $T$ being a representation implies that
\[ \Delta(T_{ij}) = \sum_{k=1}^n T_{ik}\otimes T_{kj} \qquad (1\leq i,j\leq n), \]
see before Lemma~2.6 in \cite{soltan}.  Again, we identify $M(C^*_r(G))$ with a
$*$-subalgebra of $VN(G)$.  Now, $VN(G)$ becomes an $A(G)$-module, and we have that
\[ \ip{a\cdot T}{b} = \ip{T}{ab} = \ip{\Delta(T)}{a\otimes b} \qquad (a,b\in A(G), T\in VN(G)). \]
Hence
\[ a\cdot T_{ij} = \sum_{k=1}^n \ip{T_{ik}}{a} T_{kj} \qquad (1\leq i,j\leq n, a\in A(G)), \]
showing in particular that the map $A(G)\rightarrow VN(G), a\mapsto a\cdot T_{ij}$
is finite-rank.  By Chou's work, \cite[Proposition~2.3]{chou}, this implies that
$T_{ij}\in\Trig(\hat G)$.  It is easy to see (compare \cite[Lemma~2.6]{soltan}) that
the set of matrix elements of $T$ agrees with the linear span of $\{ T_{ij} : 
1\leq i,j\leq n\}$, so we conclude that each matrix element of $T$ is a member of
$\Trig(\hat G)$, as required.
\end{proof}

Instead of working with $C^*_r(G)$, one could start with the full C$^*$-algebra
$C^*(G)$.  S{\o}ltan shows that $\AP(C^*(G))$ is also equal to the closure of
the span of $\{\lambda(s) : s\in G\}$, but this time inside $M(C^*(G))$.

The following tells us that, informally, there is no ``bigger'' compact quantum
group sitting inside $VN(G)$ than $C^*_\delta(G)$.

\begin{corollary}
Let $G$ be a locally compact group, and let $\mc A \subseteq VN(G)$ be a unital
$*$-subalgebra such that $\Delta$ restricts to give a map $\mc A \rightarrow
\mc A \otimes \mc A$, making $(\mc A,\Delta)$ into a compact quantum group.
Then $\mc A \subseteq C^*_\delta(G)$.
\end{corollary}
\begin{proof}
Let $A$ be the subspace of $\mc A$ formed by taking the linear span of matrix
elements of finite-dimensional, unitary representations of $\mc A$.  By
\cite[Theorem~2.2]{woro1} we know that $A$ is a dense $*$-subalgebra of $\mc A$.
Examining the above proof, we see immediately that $A \subseteq \Trig(\hat G)$,
from which it follows that $\mc A \subseteq C^*_\delta(G)$.
\end{proof}

If we replaced $VN(G)$ by $M(C^*_r(G))$, then the above corollary would immediately
follow from the definition of $\mf b C^*_r(G)$ as a universal object behaving
as a compactification.





\section{Almost periodic elements}

Let $\mc A$ be a Banach algebra, and turn $\mc A'$ into an $\mc A$-bimodule in the usual way,
\[ \ip{a\cdot\mu}{b} = \ip{\mu}{ba}, \quad
\pi{\mu\cdot a}{b} = \ip{\mu}{ab} \qquad (a,b\in\mc A,\mu\in\mc A'). \]
For $\mu\in\mc A'$, let $\mc R_\mu:\mc A\rightarrow\mc A'$ be the map $\mc R_\mu(a) = a\cdot\mu$.
When $\mc R_\mu$ is a \emph{compact} operator, we say that $\mu$ is \emph{almost periodic},
written $\mu\in\ap(\mc A)$.  We can show that $\ap(\mc A)$ is a closed $\mc A$-submodule of
$\mc A'$.

We similarly turn $\mc A''$ into an $\mc A$-bimodule.  We define an action of $\mc A''$ on
$\mc A'$ by
\[ \ip{\Phi\cdot\mu}{a} = \ip{\Phi}{\mu\cdot a}, \quad
\ip{\mu\cdot\Phi}{a} = \ip{\Phi}{a\cdot\mu} \qquad (a\in\mc A, \mu\in\mc A', \Phi\in\mc A''). \]
Then we define two bilinear maps $\aone,\atwo$ on $\mc A''$ by
\[ \ip{\Phi\aone\Psi}{\mu} = \ip{\Phi}{\Psi\cdot\mu}, \quad
\ip{\Phi\atwo\Psi}{\mu} = \ip{\Psi}{\mu\cdot\Phi} \qquad (\Phi,\Psi\in\mc A'', \mu\in\mc A'). \]
Then $\aone$ and $\atwo$ are algebra products, called the \emph{Arens products}.  They
both extend the product on $\mc A$ in the sense that, for $a\in\mc A$ and $\Phi\in\mc A''$,
 $a\cdot\Phi = \kappa_{mc A}(a)\aone\Phi = \kappa_{\mc A}(a)\atwo\Phi$, and similarly on the
right.  See \cite[???]{Dales} or \cite[???]{palmer1} for further details.

The following result essentially appears a number of times in the literture, but for our
purposes, is well treated by Lau in \cite[Theorem~5.8]{lau}.

\begin{proposition}\label{ap_for_banalg}
Let $\mc A$ be a Banach algebra, and let $X\subseteq\mc A'$ be a closed $\mc A$-submodule.
Then the following are equivalent:
\begin{enumerate}
\item $X\subseteq\ap(\mc A)$;
\item Treating $X'$ as a quotient of $A''$, both Arens products drop to well-defined
algebra products on $X'$, these products agree, and the resuling product is jointly
weak$^*$-continuous on bounded subsets.
\end{enumerate}
\end{proposition}

Let $G$ be a locally compact group.  We say that $f\in C^b(G)$ is \emph{almost periodic}
if the set $\{ f_s : s\in G \}$ is relatively compact in $C^b(G)$, where $f_s(t) = f(ts)$ for
$t\in G$.  Using bounded approximate identities, it is easy (compare with \cite{ulger}) to
show that $\ap(L^1(G)) = \ap(G)$.  We know that $\ap(G)$ is a closed $*$-subalgebra of
$C^b(G)$, with spectrum $G^\ap$ say.  Then $G$ is dense in $G^\ap$, and extending the product
from $G$ to $G^\ap$ turns $G^\ap$ into a compact group.  Furthermore, $G^\ap$ then agrees with
the Bohr compactification of $G$.  See \cite{BJM} for further details.

As $A(G)$ can be thought of as being the ``dual'' to $L^1(G)$ (hence extending the abelian case,
when $A(G) = L^1(\hat G)$), it is not surprising that $\ap(A(G))$ has been studied.  It is
customary to write $\ap(\hat G)$ for $\ap(A(G))$.  See the references \cite[Section~7]{DR},
\cite{gran1}, \cite{lau}, \cite{hu}, \cite{must}, and especially \cite{chou} for further
details.  It would be fair to say that, for a general locally compact group $G$, the above
proposition is about all that is known about $\ap(\hat G)$ (though see below for our discussion
of invariant states).

When $G$ is abelian, by applying Fourier transforms, we can show that $\ap(\hat G) = C^*_\delta(G)
= \mf b C^*_r(G)$.  Similarly, when $G$ is discrete and amenable, we have $\ap(\hat G) =
\mf b C^*_r(G)$ (this follows from inspecting \cite[Proposition~2]{gran1}).  We do not know,
in general, if $\ap(\hat G)$ need be a C$^*$-algebra.  Chou defines a group $G$ to have the
\emph{dual Bohr approximation property} if $\ap(\hat G) = C^*_\delta(G) = \mf bC^*_r(G)$.
In \cite{chou}, and in Rinder's correction \cite{rindler}, it is shown that there exist
compact groups $G$ which fail to have the dual Bohr approximation property.  Indeed, the idea
is as follows.  Let $E\in VN(G)$ be the rank-one operator defined by
\[ E(f) = \Big(\int_G f \Big) 1_G \qquad (f\in L^2(G)), \]
where $1_G$ is the constant function on $G$.  Then it is possible for both
$\mc R_E$ to be compact, that is, $E\in\ap(\hat G)$, and for us to have that
$E\not\in C^*_\delta(G)$.



\subsection{Approximation properties}

We recall the classical notion of the \emph{Banach space approximation property}
(see, for example, \cite[Section~1.e]{lt}).  Recall that a Banach space $E$ has the
approximation property if, for any Banach space $F$, any compact operator $T:F\rightarrow E$
can be norm-approximated by finite-rank operators.  In general, an operator which can be
approximated by finite-ranks is said to be \emph{approximable}.

The Banach space $L^\infty(G)$ always has the approximation property, so in particular,
asking that $\mc R_\mu$, for $\mu\in L^\infty(G)$, be compact is the same as asking if we can
approximate $\mc R_\mu$ be finite-rank maps.  However, we know that $VN(G)$ has the approximation
property if and only if $VN(G)$ is \emph{amenable} as a Banach algebra, which in turn is
equivalent to $VN(G)$ being isomorphic to an algebra of the from
\[ \bigoplus_{k=1}^n \mathbb M_{n_k} \otimes C(X_k). \]
It follows that this is equivalent to $G$ being abelian by finite.  See \cite[Section~6.1]{runde}
for details.  In conclusion, for general groups $G$, we have no reason to believe that
$\mc R_\mu$ being compact is equivalent to $\mc R_\mu$ being approximable.

So perhaps a better definition of $\mu$ being ``almost periodic'' would be to ask for
$\mc R_\mu$ to be approximable.  We shall show below that if we do ask this, but in the
category of \emph{operator spaces}, then we get $\mf b C^*_r(G)$, as we might hope.  In the 
remainder of this section, we shall study some stronger forms of ``approximable'' and show that
each of these gives rise to a notion of ``almost periodic'' leading back to $\mf b C^*_r(G)$,
at least when the group $G$ is amenable.

Firstly, define $\ap^{(1)}(\hat G)$ to be the collection of $T\in VN(G)$ such that $\mc R_T$ can
be norm-approximated by maps of the form $\mc R_S$, where $S\in VN(G)$ and $\mc R_S$ is
finite-rank.  It follows immediately from Chou's work that $\ap_1(\hat G)=\mf bC^*_r(G)$.
Indeed, by \cite[Proposition~2.3]{chou}, we have that $\mc R_S$ is finite-rank if and only if
$S\in\Trig(\hat G)$.  If $G$ is amenable, then $\|\mc R_S\| = \|S\|$ for all $S\in VN(G)$.
So immediately, $\ap_1(\hat G) = \mf b C^*_r(G)$, when $G$ is amenable.

Define $\ap^{(2)}(\hat G)$ to be the collection of $T\in VN(G)$ such that $\mc R_T$ can be
norm-approximated by finite-rank $A(G)$-module maps $\theta:A(G)\rightarrow VN(G)$.
That is, maps $\theta$ such that $\theta(ab) = a\cdot\theta(b)$ for $a,b\in A(G)$.  If $A(G)$
has a bounded approximate identity, that is, $G$ is amenable, then a standard argument shows
that $\theta$ is of the form $\mc R_S$ for some $S\in VN(G)$.  Infact, the following proposition
shows that this is true in general, and so $\ap_2(\hat G) = \ap_1(\hat G)$.

\begin{proposition}
A finite-rank map $\theta:A(G)\rightarrow VN(G)$ is a module map if and only if
$\theta=\mc R_S$ for some $S\in\Trig(\hat G)$.
\end{proposition}
\begin{proof}

\end{proof}





\section{Operator-space structures}

There is now ample evidence that when considering the Fourier algebra $A(G)$, if we view
$A(G)$ as a \emph{completely-contractive Banach algebra} with its canonical operator-space
structure, then $A(G)$ behaves much more closely to being the ``dual'' of $L^1(G)$, as compared
to viewing $A(G)$ as simply a Banach algebra.  See, for example, Ruan's original paper
\cite{ruan} or Runde's survey article \cite{runde1}.

We shall assume standard results about operator spaces, as found in \cite{ER} or \cite{pisier}.
The important fact for us is that $A(G) \proten A(G) = A(G\times G)$, where $\proten$ denotes
the \emph{operator space projective tensor product}.  Let $T\in VN(G)$, and consider the
map $\mc R_T$, now treated as a completely bounded map $A(G)\rightarrow VN(G)$.  Now,
$\mc{CB}(A(G),VN(G)) = (A(G)\proten A(G))'$, and we see that
\[ \ip{\mc R_T(a)}{b} = \ip{T}{ab} = \ip{\Delta(T)}{a\otimes b} \qquad(a,b\in A(G), \]
so $R_T(a) = (a\otimes\id)\Delta(T)$.  As $A(G)\proten A(G) = A(G\times G)$, we have
\[ \|\mc R_T\|_{cb} = \sup\{ |\ip{\Delta(T)}{a}| : a\in A(G\times G), \|a\|\leq 1 \}
= \|\Delta(T)\| = \|T\|. \]

We start by showing that operator space versions of $\ap^{(1)}$ and $\ap^{(2)}$ allow us to
remove the condition that $G$ be amenable.

Let $\ap^{(1)}_{cb}(\hat G)$ be the collection of $T\in VN(G)$ such that $\mc R_T$ can be
approximated, in the completely bounded norm, by maps of the form $\mc R_S$ such that
$\mc R_S$ is finite rank.  Again, this means that $S\in\Trig(\hat G)$.  However, we now have
that $\|\mc R_T-\mc R_S\|_{cb} = \|T-S\|$, so we immediately conclude that $T\in C^*_\delta(G)$.
Hence $\ap^{(1)}_{cb}(\hat G) = \mf bC^*_r(G)$.

Similarly, if we define $\ap^{(2)}_{cb}(\hat G)$ to be the collection of $T\in VN(G)$ such
that $\mc R_T$ can be approximated, in the completely bounded norm, but finite-rank module
maps, then as above, $\ap^{(2)}_{cb}(\hat G) = \ap^{(1)}_{cb}(\hat G) = \mf bC^*_r(G)$.

These are very strong approximation properties.  A much weaker notion is the following.
Define $\ap_{cb}(\hat G)$ be the the collection of $R\in VN(G)$ such that $\mc R_T$ can
be approximated, in the completely bounded norm, by arbitrary finite-rank maps from $A(G)$ to
$VN(G)$.  From \cite[Proposition~8.1.2]{ER}, it follows that $T\in\ap_{cb}(\hat G)$ if and
only if $\Delta(T) \in VN(G) \inten VN(G)$, where here $\inten$ denotes the operator space
\emph{injective tensor product}.  This agrees with the minimal (or spacial) C$^*$-tensor
product (see \cite[Proposition~8.1.6]{ER}).

\begin{theorem}
Let $G$ be a discrete group.  Then $\ap_{cb}(\hat G) = \mf b C^*_r(G) = C^*_r(G)$.
\end{theorem}
\begin{proof}
Let $(e_s)_{s\in G}$ be the standard orthonormal basis of $\ell^2(G)$.  Each $T\in VN(G)$
is uniquely determined by the vector $\tau = T(e_{e_G})\in\ell^2(G)$, as $T(x) = \tau * x$
for $x\in\ell^2(G)$, namely the convolution of $\tau$ with $x$.  Under this, we identify
$VN(G)$ with the collection of vectors $\tau\in\ell^2(G)$ such that convolution by $\tau$ on
the left gives a bounded operators on $\ell^2(G)$.  Under this identification, $\Trig(G)$
is simply the linear span of $(e_s)$ in $\ell^2(G)$.

Similarly, we identify $VN(G\times G)$ with a subspace of $\ell^2(G\times G) = \ell^2(G)
\otimes \ell^2(G)$.  We can check that if $T\in VN(G)$ and $\tau = T(e_{e_G})$, then
\[ \Delta(T)(e_{(e_G,e_G)}) = ( \tau_s \delta_{s,t} )_{(s,t)\in G\times G}, \]
where $\delta_{s,t}$ is the Kronecker delta.

Define $U:\ell^2(G)\rightarrow\ell^2(G\times G)$ and
$V:\ell^2(G\times G)\rightarrow\ell^2(G)$ by $U(e_s) = e_{(s,s)}$ and $V(e_{(s,t)})
= \delta_{s,t} e_s$ for $s,t\in G$.
Define a map $\theta:VN(G\times G) \rightarrow \ell^2(G)$ as follows.  For $R\in VN(G\times G)$,
let $\rho = R(e_{(e_G,e_G)}) \in \ell^2(G\times G)$.  Then define
\[ \theta(R) = \big( \rho_{s,s} \big)_{s\in G} \in \ell^2(G). \]
Notice that
\[ VRU(e_s) = VR(e_{(s,s)}) = \sum_{t,r\in G} \rho_{(t,r)} V(e_{(ts,rs)})
= \sum_{t,r\in G} \rho_{(t,r)} \delta_{t,r} e_{ts}
= \sum_{t\in G} \rho_{(t,t)} e_{ts} = \theta(R) * e_s. \]
Consequently, convolution on the left by $\theta(R)$ is a bounded operator, namely
$VRU$, and so $\theta(R)$ induces a member of $VN(G)$.
Notice that
\[ V\Delta(T)U = T \qquad (T\in VN(G)). \]

If $T,S\in VN(G)$, say $\tau = T(e_{e_G})$ and $\sigma = S(e_{e_G})$, then
\[ \theta(T\otimes S) = \big( \tau_s \sigma_s \big)_{s\in G}. \]
The Cauchy-Schwarz inequality shows that $\theta(T\otimes S)\in\ell^1(G)\subseteq\ell^2(G)$,
and so $V(T\otimes S)U$ is in the closure of $\Trig(\hat G)$, that is, in $C^*_r(G)$.
By continuity, we conclude that $V R U \in C^*_r(G)$ for any $R\in VN(G) \inten VN(G)$.
Hence, if $T\in VN(G)$ is such that $\Delta(T)\in VN(G)\inten VN(G)$, then
$T = V\Delta(T)U \in C^*_r(G)$, as required.
\end{proof}

We shall have to work harder to prove this result for arbitrary locally compact groups $G$.

One approach would be the following.  Let $\Lambda$ be the collection on compact,
symmetric neighbourhoods of $e_G$.  For $T\in VN(G)$, let
\[ x_K = T(\chi_K) |K|^{-1} \qquad (K\in\Lambda). \]
Then $(x_K)$ is a (in general unbounded) net in $L^2(G)$.  However, the net
$(\chi_K |K|^{-1})_{K\in\Lambda}$ is a bounded net in $L^1(G)$ which is an approximate
identity.  In particular, for $f\in L^2(G)$, we have that $f = \lim_K |K|^{-1}
\chi_K * f$.  As $VN(G)$ commutes with convolution on the right, we have that
\[ T(f) = \lim_K T(\chi_K*f) |K|^{-1} = \lim_K T(\chi_K)*f |K|^{-1}
= \lim_K x_K * f \qquad (f\in L^2(G)), \]
where the limit is in norm.

Thus $(x_K)$ determines $T$; in fact, if $\psi$ is the left Plancherel weight, then
$\psi(T^*T) = \limsup_K \|x_K\|_2^2$.  For $T\in VN(G)$ and $f,g\in L^2(G)$, we have
\[ (T^*(f)|g) = (f|T(g)) = \lim_K (f|x_K*g) = \lim_K (x_K^* * f|g). \]
Hence $|K|^{-1} T^*(\chi_K)$ might not be $x_K^*$, but the net $(x_K^*)_{K\in\Lambda}$
does induce the operator $T^*$.  Here the involution is that on $L^1(G)$,
\[ x_K^*(s) = \Delta(s^{-1}) \overline{ x_K(s) } \qquad (s\in G). \]
As this is only defined upon $L^2(G) \cap L^1(G)$, we technically have to perturb
each $x_K$ so that it is integrable.

The problem is then to reconstruct $(x_K)$ from $\Delta(T)$ in some way.  For example,
\[ x_K\otimes f = |K|^{-1} T(\chi_K)\otimes f = |K|^{-1} W\Delta(T)W^*(\chi_K\otimes f)
\qquad (f\in L^2(G)). \]
Hence, for $f\in L^2(G)$, we have that
\begin{align*} (x_K|f) &= \lim_L |L|^{-1} (x_K\otimes\chi_L | f\otimes \chi_L) \\
&= \lim_L |L|^{-1} |K|^{-1} \big( \Delta(T)W^*(\chi_K\otimes \chi_L) |
   W^*(f\otimes\chi_L) \big).
\end{align*}
For each $L\in\Lambda$, let $e_L = \chi_L |L|^{-1/2}$, so that $\|e_L\|_2=1$.  Then,
for any $R\in\mc B(L^2(G\times G))$, the following limit exists
\[ \phi_{K,f}(R) = \lim_{L\in\Lambda} |K|^{-1} \big( RW^*(\chi_K\otimes e_L) |
W^*(f\otimes e_L) \big), \]
so that $\phi_{K,f}\in\mc B(L^2(G\times G))'$ with $\|\phi_{K,f}\|\leq |K|^{-1/2}
\|f\|_2$.  We then have that
\[ (x_K|f) = \ip{\phi_{K,f}}{\Delta(T)} \qquad (f\in L^2(G)). \]

\textbf{How are we going to do this?}

\noindent\textbf{Idea:}
We can do the above for $VN(G\times G)$ as well, of course.  The idea, copying what we
did in the discrete case, would be to find some \emph{bounded} operation which takes
a ``convolution family'' given by $VN(G\times G)$ and gives back a ``convolution
family'' inducing a member of $VN(G)$.  We want this to be an inverse to $\Delta$, and
which hopefully does interesting things to $VN(G) \otimes VN(G)$.




\begin{thebibliography}{99}
\normalsize
\newcommand{\bibbook}[3]{\textsc{#1}, \emph{#2}, (#3).}
\newcommand{\bibpaper}[6]{\textsc{#1}, `#2', \emph{#3} #4 (#5) #6.}
\newcommand{\bibpreprint}[2]{\textsc{#1}, `#2', preprint.}

\bibitem{BMT} \bibpaper{E. B\'edos, G.\,J. Murphy, L. Tuset}
   {Co-amenability of compact quantum groups}
   {J. Geom. Phys.}{40}{2001}{130--153}

\bibitem{BJM} \bibbook{J.\,F. Berglund, H.\,D. Junghenn, P. Milnes}
   {Analysis on semigroups.  Function spaces, compactifications, representations}
   {John Wiley \& Sons, Inc., New York, 1989}

\bibitem{chou} \bibpaper{C. Chou}
   {Almost periodic operators in ${\rm VN}(G)$}
   {Trans. Amer. Math. Soc.}{317}{1990}{229--253}

\bibitem{Dales} \bibbook{H.\,G. Dales}
   {Banach algebras and automatic continuity}
   {Clarendon Press, Oxford, 2000}

\bibitem{DR} \bibpaper{C.\,F. Dunkl, D.\,E. Ramirez}
   {Weakly almost periodic functionals on the Fourier algebra}
   {Trans. Amer. Math. Soc.}{185}{1973}{501--514}

\bibitem{DR2} \bibpaper{C.\,F. Dunkl, D.\,E. Ramirez}
   {Existence and nonuniqueness of invariant means on {${\mc L}\sp{\infty }(\hat G)$}}
   {Proc. Amer. Math. Soc.}{32}{1972}{525--530}

\bibitem{ER} \bibbook{E.\,G. Effros, Z.-J. Ruan}
   {Operator spaces}
   {Oxford University Press, New York, 2000}

\bibitem{eymard} \bibpaper{P. Eymard}
   {L'alg\`ebre de Fourier d'un groupe localement compact}
   {Bull. Soc. Math. France}{92}{1964}{181--236}

\bibitem{gran1} \bibpaper{E.\,E. Granirer}
   {Weakly almost periodic and uniformly continuous functionals
   on the Fourier algebra of any locally compact group}
   {Trans. Amer. Math. Soc.}{189}{1974}{371--382}

\bibitem{hu} \bibpaper{Z. Hu}
   {Open subgroups of $G$ and almost periodic functionals on $A(G)$}
   {Proc. Amer. Math. Soc.}{128}{2000}{2473--2478}

\bibitem{kus} \bibpaper{J. Kustermans}
   {Locally compact quantum groups in the universal setting.}
   {Internat. J. Math.}{12}{2001}{289--338}

\bibitem{kv} \bibpaper{J. Kustermans, S. Vaes}
   {Locally compact quantum groups}
   {Ann. Sci. \'Ecole Norm. Sup. (4)}{33}{2000}{837--934}

\bibitem{lance} \bibbook{E.\,C. Lance}
   {Hilbert $C\sp *$-modules. A toolkit for operator algebraists}
   {London Mathematical Society Lecture Note Series, 210.
   Cambridge University Press, Cambridge, 1995.}

\bibitem{lau} \bibpaper{A.\,T.-M. Lau}
   {Uniformly continuous functionals on the Fourier algebra of any locally compact group}
   {Trans. Amer. Math. Soc.}{251}{1979}{39--59}

\bibitem{lt} \bibbook{J. Lindenstrauss, L. Tzafriri}
   {Classical Banach spaces. I.}
   {Springer-Verlag, Berlin-New York, 1977}

\bibitem{maes} \bibpaper{A. Maes, A. Van Daele}
   {Notes on compact quantum groups}
   {Nieuw Arch. Wisk. (4)}{16}{1998}{73--112}

\bibitem{murphy} \bibpaper{G.\,J. Murphy, L. Tuset}
   {Aspects of compact quantum group theory}
   {Proc. Amer. Math. Soc.}{132}{2004}{3055--3067}

\bibitem{must} \bibpaper{H.\,S. Mustafayev}
   {Almost periodic functionals on some class of Banach algebras}
   {Rocky Mountain J. Math.}{36}{2006}{1977--1997}

\bibitem{ped} \bibbook{G.\,K. Pedersen}
   {$C\sp{*} $-algebras and their automorphism groups}
   {Academic Press, Inc., London-New York, 1979}

\bibitem{pisier} \bibbook{G. Pisier}
   {Introduction to operator space theory}
   {Cambridge University Press, Cambridge, 2003}

\bibitem{renaud} \bibpaper{P.\,F. Renaud}
   {Invariant means on a class of von {N}eumann algebras}
   {Trans. Amer. Math. Soc.}{170}{1972}{285--291}

\bibitem{rindler} \bibpaper{H. Rindler}
   {On weak containment properties}
   {Proc. Amer. Math. Soc.}{114}{1992}{561--563}

\bibitem{ruan} \bibpaper{Z.-J. Ruan}
   {The operator amenability of $A(G)$}
   {Amer. J. Math.}{117}{1995}{1449--1474}

\bibitem{runde} \bibbook{V. Runde}
   {Lectures on amenability}
   {Lecture Notes in Mathematics, 1774. Springer-Verlag, Berlin, 2002}

\bibitem{runde1} \bibpaper{V. Runde}
   {Applications of operator spaces to abstract harmonic analysis}
   {Expo. Math.}{22}{2004}{317--363}

\bibitem{soltan} \bibpaper{P.\,M. So\l tan}
   {Quantum Bohr compactification}
   {Illinois J. Math.}{49}{2005}{1245--1270}

\bibitem{tak1} \bibbook{M. Takesaki}
   {Theory of operator algebras. I.}
   {Springer-Verlag, New York-Heidelberg, 1979}

\bibitem{ulger} \bibpaper{A. \"Ulger}
   {Continuity of weakly almost periodic functionals on $L\sp 1(G)$}
   {Quart. J. Math. Oxford Ser. (2)}{37}{1986}{495--497}

\bibitem{vn1} \bibpaper{J. v.\,Neumann}
   {Almost periodic functions in a group. I.}
   {Trans. Amer. Math. Soc.}{36}{1934}{445--492}

\bibitem{woro1} \textsc{S.\,L. Woronowicz},
   `Compact quantum groups', in \emph{Symetries quantiques} (Les Houches, 1995),
   845--884, North-Holland, Amsterdam, 1998. 

\end{thebibliography}

\bigskip\noindent\emph{Email:} \texttt{matt.daws@cantab.net}

\end{document}

